\achapter{42}{Answers and Hints to Selected Exercises} \label{sec:answers}

\begin{multicols}{2}
%1
Section \ref{sec:intro_linear_systems}
\obe
\item 
\ba
\item Consider the equations 
\[b_1x_1+b_2x_2 + b_3x_3 = s,\]
and
\[k(b_1x_1+b_2x_2 + b_3x_3) = ks.\]

\item Similar to part (a)

\ea


\item A cup of coffee costs \$$1.95$, a muffin costs \$$2.05$, and a scone costs \$$2.15$. 


\item
\ba
\item 
\begin{align*}
x_1 &= \frac{x_2+200+0+0}{4} \\
x_2 &= \frac{x_1+x_3+200+0}{4} \\
x_3 &= \frac{x_2+0+400+0}{4}
\end{align*}

\item $x_1 = 75$, $x_2 = 100$ and $x_3 = 125$. 

\ea


\item 
\ba
	\item 
\begin{alignat*}{5}
{}x_1 	&{}+{}	&{}x_2	&{}+{}	&{}x_3	&= &\ 0&{}  \\ 
{}x_1 	&{}+{}	&{}x_2	&{}+{}	&{}x_3	&= &\ 1&{.}   
\end{alignat*}

	\item Impossible. 
	
	\item Impossible. 
	
	\item 
\begin{alignat*}{5}
{}x_1 	&{}+{}	&{}x_2	&{}+{}	&{}x_3	&= & \ 0&{}  \\ 
{2}x_1 	&{}+{}	&{2}x_2	&{}+{}	&{2}x_3	&= & \ 0&{.}   
\end{alignat*}

	\ea
			
\item 
\ba
\item This system has the solution $x_1 =  \frac{h-10}{3h-5}$ and $x_2 = \frac{5}{3h-5}$ as long as $h \neq \frac{5}{3}$. 

\item When $h = \frac{5}{3}$, $x_1 =  2-\frac{5}{3}x_2$ and $x_2$ is free. 

\ea

\item %True/False Questions
\begin{enumerate}[label=(\alph*), leftmargin=1\parindent]
	\item F 
	\skipitems{1}
	\item F
	\skipitems{1}
	\item  F
	\end{enumerate}
		
	
	
	
\oee

%2
\hspace{-0.25in} Section \ref{sec:matrix_representation}

\obe
\item 
	\ba
	\item $h=6$

	\item $h=6$ and $k \neq -2$. 

	\item $h=6$ and $k=-2$. 

	\ea



\item 
\begin{alignat*}{5}
{}x_1 	&{}+{}	&{}x_2	&{}{}		&{}		&{}={}	&\ {}&1&{}\\
{-}x_1	&{}+{}	&{}x_2	&{}{}		&{}		&{}={} 	&\ {-}&3&{} \\
{}x_1		&{}+{}	&{}x_2	&{}+{}	&{}x_3	&{}={}	& \ {}&1&{.}
\end{alignat*}


\item Student 2's hunch is correct. 

\oee

\be
\item [6.]
	\begin{enumerate}[label=(\alph*), leftmargin=1\parindent]
	\item F
	\skipitems{1}
	\item T 
	\skipitems{1}
	\item T
	\skipitems{1}
	\item F 
	\skipitems{1}
	\item F
	\skipitems{1}
	\item F
	\skipitems{1}
	\end{enumerate}
\ee

%3
\hspace{-0.25in} Section \ref{sec:row_echelon_forms}

\obe
\item Augmented matrix: \\
$\left[ \begin{array}{ccr|c} 1&1&-1&4 \\ 1&2&2&3 \\ 2&3&-3&11 \end{array} \right]$, \\
solution $x_1 = 1$, $x_2=2$, and $x_3 = -1$ 

\item Row operations show this. 

\item $x_1 = -4-6x_3$, $x_2 = -3-4x_3$, $x_3$ is free, $x_4 = 3$

\item 
\ba
\item Yes

\item $\left[ \begin{array}{cc|c} \blacksquare&*&* \\ 0&\blacksquare&* \end{array} \right]$, $\left[ \begin{array}{cc|c} 0&\blacksquare&* \\ 0&0&* \end{array} \right]$

\ea

\item This is not possible.


\item $\left[ \begin{array}{cc} \blacksquare&*\\0&* \end{array} \right]$ and $\left[ \begin{array}{cc} 0&*\\0&0 \end{array} \right]$.

\oee

\be
\item[12]
\begin{enumerate}[label=(\alph*), leftmargin=1\parindent]
	\item T
\skipitems{1}	

	\item T
\skipitems{1}

	\item  F
\skipitems{1}


	\item T
	\skipitems{1}

	\item F 
\skipitems{1}

	\item F

	\end{enumerate}
\ee

%4
\hspace{-0.25in} Section \ref{sec:vector_representation}


\obe
\item $\vw = \left(-\frac{9}{4}\right) \vu + \left(\frac{7}{4}\right) \vv$

\item Only when $w_1 = 2w_2 - 3w_3$. $0 \vu + 0\vv = \vzero$

\item 
\ba
\item We cannot make the desired solution. 

\item We can make the desired chemical solution with $\frac{3}{4}$ of solution $\vv_1$ for every $\frac{1}{4}$ of solution $\vv_2$. 

\item If $\vw = \left[ \begin{array}{c} w_1\\w_2\\w_3\end{array}\right]$ can be made from our original two chemical solutions, then $w_3+\frac{3}{23}w_1-\frac{7}{23}w_2 = 0$.  

\ea
	

\item 
\ba
\item $\Span\{\vv_1\}$ is the line in $\R^2$ through the origin and the point $(1,1)$. 
\item $\Span\{\vv_1,\vv_3\}$ is the plane in $\R^3$ through the origin and the points $(1,1,1)$ and $(2,0,1)$. 
\ea

\item Yes. 

\item \begin{enumerate}[label=(\alph*), leftmargin=1\parindent]
\item  F
\skipitems{1}
\item  T
\skipitems{1}
\item  T
\skipitems{1} 
\item  F
\skipitems{1}
\item T
\skipitems{1}
\item  T
\skipitems{1}
\item F
\end{enumerate}

\oee

%5
\hspace{-0.25in} Section \ref{sec:matrix_vector}

\obe
\item  The matrix-vector form of the system is $A \vx = \vb$, where $\vx = \left[ \begin{array}{c} x_1\\x_2\\x_3\\x_4 \end{array} \right]$. 


\item  The system corresponding to this matrix-vector equation is 
\begin{alignat*}{4}
{2}x_1 	&{}+{} 	&{3}x_2 	&{}+{}	&{4}x_3 		&{}={}	&{}4&{} \\
{}x_1 	&{}-{} 	&{2}x_2	&{}+{}	&{3}x_3 		&{}={}	&{-}6&{.}
\end{alignat*}
The solution to the system is 
\begin{align*}
x_1 &= - \frac{17}{7}x_3 -  \frac{10}{7} \\
x_2 &= \frac{2}{7}x_3 +  \frac{16}{7} \\
x_3 &\text{ is free}.
\end{align*}


\item Both methods give 
\[\left[ \begin{array}{cc} a & b\\ c& d \end{array} \right]\left[ \begin{array}{c} x_1 \\ x_2 \end{array} \right] = \left[ \begin{array}{c} x_1a+x_2b \\ x_1c+x_2d \end{array} \right].\]


\item 
\ba
\item  $b_3$ 

\item If $b_3 = 9$, then $a = 2$. 

\ea


\item $A\left[ \begin{array}{r} -1 \\ 6 \\ -5 \end{array} \right] =  \left[ \begin{array}{r} 3 \\ -5 \end{array} \right]$.



\item The general solution to the system is 
\[\left[ \begin{array}{c} x\\y\\z \end{array} \right]  = y \left[ \begin{array}{c} 2\\1\\0 \end{array} \right] + z \left[ \begin{array}{r} -1\\0\\1 \end{array} \right] +  \left[ \begin{array}{c} 3\\0\\0 \end{array} \right].\] 
The solution is the plane in $\R^3$ through the points $(3,0,0)$, $(5,1,0)$, and $(2,0,1)$. 

\item $\vx = \left[ \begin{array}{r} 2\\-1\\-1 \end{array} \right]$ 

\oee

\be
\item[14.] 
\begin{enumerate}[label=(\alph*), leftmargin=1\parindent]
\item F
\skipitems{1}
\item T
\skipitems{1} 
\item T
\skipitems{1}
\item T
\skipitems{1}
\item T
\skipitems{1}
\item T
\end{enumerate}
\ee

%6
\hspace{-0.25in} Section \ref{sec:independence}

\obe

\item Linearly dependent. Change the last entry in $\vv_3$ to a $0$.

\item Only two of the solutions are necessary. 

\item 
	\ba
	\item  $\{\vv_1, \vv_2, \vv_3\}$ 

	\item $\vu = \vv_1 - 2 \vv_2 + 3 \vv_3$

	\ea
	
\item This is not possible. 

\item
\begin{enumerate}[label=(\alph*), leftmargin=1\parindent]
\item F
\skipitems{1}
\item T
\skipitems{1}
\item T
\skipitems{1}
\item T
\skipitems{1}
\item T
\skipitems{1}
\item T
\end{enumerate}


\oee

%7
\hspace{-0.25in} Section \ref{sec:matrix_transformations}

\obe 
\item Ff $\vx = \left[ \begin{array}{c} x_1\\x_2\\x_3 \end{array} \right]$, then 
$T(\vx) =  \left[ \begin{array}{c} x_1+2x_2+x_3 \\ x_1-3x_3 \end{array} \right]$.

\item $\left[ \begin{array}{r} 12 \\ -11 \end{array} \right]$.

\item $A = \left[ \begin{array}{cr} 2&-1\\1&3 \end{array} \right]$

\item $T$ defined by $T(\vx) = \left[ \begin{array}{ccc} 1&0&0 \\ 0&1&0 \\ 0&0&1 \\0&0&0 \end{array} \right]\vx$ 

\item Not possible.

\item $L= \vv+c\vw$, $T(L) = T(\vv+c\vw) = T(\vv) + cT(\vw)$ 

\oee

\be
\item[12.]
\begin{enumerate}[label=(\alph*), leftmargin=1\parindent]
\item  F
\skipitems{1}
\item F
\skipitems{1}
\item T
\skipitems{1}
\item F
\skipitems{1}
\item F
\skipitems{1}
\item F
\skipitems{1}
\item T
\skipitems{1}
\item T
\end{enumerate}

\ee

%8
\hspace{-0.25in} Section \ref{sec:matrix_operations}

\obe

\item 
\ba
\item $AB = \left[ \begin{array}{cc} a&b \\ c&d \\ 0&0 \end{array} \right]$

\item $AB  = [-2 \ -2]$.

\ea

\item 
\ba
\item $X = \left[ \begin{array}{cc} 0&1\\0&0 \end{array} \right]$ 

\item There is no matrix $X$ with this property. 

\item $X = \left[ \begin{array}{cc} 2x_{21}&1+2x_{22}\\x_{21}&x_{22} \end{array} \right]$, where $x_{21}$ and $x_{22}$ can be any scalars 

\ea

\item $A^m \vv = 2^m \vv$

\item $A^2=0$ and $B^3=0$

\item 
\begin{enumerate}[label=(\alph*), leftmargin=1\parindent]
\item $[a_{ij} + b_{ij}] = [b_{ij} + a_{ij}]$ 
\skipitems{1}
\item $[a_{ij} + 0] = [a_{ij}]$
\skipitems{1}
\item $[(a+b)a_{ij}] = [aa_{ij}] + [ba_{ij}]$
\skipitems{1}
\item $[(ab)a_{ij}] = a[ba_{ij}]$   
\end{enumerate}

\item 
\begin{enumerate}[label=(\alph*), leftmargin=1\parindent]
\item Let $A^{\tr} = [a'_{ij}]$. Then $a'_{ij} = a_{ji}$ by definition of the transpose. Let $\left(A^{\tr}\right)^{\tr} = [a''_{ij}]$. Then $a''_{ij} = a'_{ji} = a_{ij}$. So the $ij$th entry of $\left(A^{\tr}\right)^{\tr}$ is the same as the $ij$th entry of $A$, and we conclude that $\left(A^{\tr}\right)^{\tr} = A$. 
\skipitems{1}	
	\item The $ij$th entry of $aA$ is $aa_{ij}$, so the $ij$th entry of $(aA)^{\tr}$ is $aa_{ji}$. But $aa_{ji}$ is also the $ij$th entry of $aA^{\tr}$. We conclude that $(aA)^{\tr} = aA^{\tr}$.
	
	\end{enumerate}


\item If $A=\left[ \begin{array}{cr} \cos(\alpha) & -\sin(\alpha) \\ \sin(\alpha) & \cos(\alpha) \end{array} \right]$, $B =  \left[ \begin{array}{cr} \cos(\beta) & -\sin(\beta) \\ \sin(\beta) & \cos(\beta) \end{array} \right]$, then  $AB$ equals \\
$\left[ \arraycolsep=1.0pt \begin{array}{cr} \cos(\alpha+\beta) & -\sin(\alpha+\beta) \\ \sin(\alpha+\beta) & \cos(\alpha+\beta) \end{array} \right]$

\oee

\be	
\item[14.] 
\begin{enumerate}[label=(\alph*), leftmargin=1\parindent]
\item  F
\skipitems{1}
\item F
\skipitems{1}
\item F
\skipitems{1}
\item T
\end{enumerate}

\ee

%9
\hspace{-0.25in} Section \ref{sec:intro_eigenvals_eigenvects}

\obe
\item
\ba
\item Eigenvector with eigenvalue $-1$.
\item Eigenvector with eigenvalue $3$. 
\item Eigenvector with eigenvalue $2$. 
\ea

\item 
\ba
\item Eigenvalue 
\item Eigenvalue
\item Not an eigenvalue
\item Not an eigenvalue
\ea


\item 
\ba
\item $a_3 = 804,000$ $s_3= 196,000$,  $a_4  = 782,400$, $s_4 = 217,600$
\item $a_{k+1}=0.9a_k+0.3s_k$, $s_{k+1}=0.1a_k+0.7s_k$
\item $\left[ \begin{array}{c} a_{k+1} \\ s_{k+1} \end{array} \right]$ equals \\
$\left[ \begin{array}{cc} 0.9 & 0.3 \\ 0.1 & 0.7\end{array} \right]  \left[ \begin{array} {c} a_k \\ s_k \end{array}\right]$
\ea

\oee

\be
\item[6.] 
\begin{enumerate}[label=(\alph*), leftmargin=1\parindent]
\item  F
\skipitems{1}
\item T
\skipitems{1}
\item T
\skipitems{1}
\item T
\skipitems{1}
\item F
\skipitems{1}
\item T
\end{enumerate}

\ee

%10
\hspace{-0.25in} Section \ref{sec:matrix_inverse}

\obe


\item $C = (I)C = C = (BA)C = BAC = B(AC) =  B(I) = B$

\item 
\ba 
\item $\left[ \begin{array}{cc} 1&k \\ 0&1 \end{array} \right]^{-1} =  \left[ \begin{array}{cr} 1&-k \\ 0&1 \end{array} \right]$ 

\item Row reduce  $[A \ | \ I_2]$

\item Row reduce $[A \ | \ I_3]$ to see that  $A^{-1} =  \left[ \begin{array}{crc} 1&-k&mk-\ell \\ 0&1&-m \\ 0&0&1 \end{array} \right]$.

\ea

\item $c \neq -4$ 


\oee

\be
\item[8.]
\begin{enumerate}[label=(\alph*), leftmargin=1\parindent]
\item T
\skipitems{1}
\item T
\skipitems{1}
\item T
\skipitems{1}
\item T
\skipitems{1}
\item F
 
\end{enumerate}

\ee

%11
\hspace{-0.25in} Section \ref{sec:IMT}

\obe
\item  $A$ is invertible as long as $\left[ \begin{array}{c} a \\ b\\ c\end{array} \right]$ is not in the plane in $\R^3$ through the origin and the points $(1,-1,1)$ and $(2,1,1)$

\item 
\begin{enumerate}[label=(\alph*), leftmargin=1\parindent]
\item  T
\skipitems{1}
\item T
\skipitems{1}
\item T
\skipitems{1}
\item T

\end{enumerate}


\oee

%12
\hspace{-0.25in} Section \ref{sec:R_n}

\obe

\item 
\ba
\item Closed under addition, not closed under multiplication by scalars, contains the zero vector. 

\item Not closed under addition, closed under multiplication by scalars, contains the zero vector. 

\item Not closed under addition, not closed under multiplication by scalars, does not contain the zero vector. 

\item Not closed under addition, closed under multiplication by scalars, contains the zero vector. 

\ea

\item The only such set is the empty set. 

\item $\R^2$. 

\item 
\ba
\item As an example, let $\vv = [2 \ 1]^{\tr}$ in $\R^2$. 
	\begin{enumerate}[label=\roman*.]
	\item $I_2$
	\item $2I_2$
	\item $[3\ve_1 \ -\ve_2]$
	\item $\left[ \frac{a}{2}\ve_1 \ b\ve_2\right] [2 \ 1]^{\tr} = a\ve_1 + b\ve_2 = [a \ b]^{\tr}$
	\end{enumerate}
	
\item $\vw_1 + \vw_2 = A_1 \vv + A_2 \vv = (A_1+A_2) \vv$, $c\vw_1 = c(A_1\vv) = A_1(c\vv)$, 
$0\vv = \vzero$.

\item $\{\vzero\}$, $\R^m$

\ea

\item $W_1 \cap W_2$ is a subspace of $\R^n$, $W_1 \cup W_2$ is not in general a subspace of $\R^n$

\item Not in general a subspace of $\R^n$. 

\item 
\begin{enumerate}[label=(\alph*), leftmargin=1\parindent]
\item  F
\skipitems{1}
\item T
\skipitems{1}
\item T
\skipitems{1}
\item F
\skipitems{1}
\item T
\skipitems{1}
\item F

\end{enumerate}

\oee

%13
\hspace{-0.25in} Section\ref{sec:null_space}

\obe
\item A basis for $\Col A$ is $\left\{ \left[ \begin{array}{c} 1 \\ 0 \\ 1 \end{array} \right], \left[ \begin{array}{c} 3 \\ 2 \\ 5 \end{array} \right]\right\}$, a basis for $\Nul A$ is 
$\left\{\left[ \begin{array}{r} -2 \\ 1 \\ 0 \\ 0 \end{array} \right], \left[ \begin{array}{r} -7 \\ 0 \\ 1 \\ 1 \end{array} \right] \right\}$.

\item $a=-3$ and $b = 0$

\item $\left[ \begin{array}{ccc} 1&0&1\\0&1&1\end{array} \right]$ 

\item Not possible 

\oee

\be
\item[8.]
\begin{enumerate}[label=(\alph*), leftmargin=1\parindent]
\item T
\skipitems{1}
\item T
\skipitems{1}
\item T
\skipitems{1}
\item F
\skipitems{1}
\item F

\end{enumerate}

\ee

%14
\hspace{-0.25in} Section \ref{sec:eigenspaces}
\obe
\item 
\ba
\item $\left\{\left[ \begin{array}{r} -1 \\ 1 \end{array} \right]\right\}$

\item $\left\{\left[ \begin{array}{r} -2 \\ 1 \end{array} \right]\right\}$

\item $\left\{\left[ \begin{array}{r} -1 \\ 1 \end{array} \right]\right\}$

\item $\left\{\left[ \begin{array}{c} 0 \\ 1 \\ 1 \end{array} \right]\right\}$

\item $\left\{\left[ \begin{array}{r} -1 \\ 2 \\ 1 \end{array} \right]\right\}$

\item $\left\{\left[ \begin{array}{r} -1 \\ 1 \\ 0 \end{array} \right], \left[ \begin{array}{r} -2 \\ 0 \\ 1 \end{array} \right]\right\}$

	\ea


\item $a=0$, $b=4$

\item ~

\ba
\item Eigenvalue $0$, eigenspace $\R^2$

\item Eigenvalue $0$, eigenspace $\R^n$ 

\ea

\oee

\be
\item[6.]
\begin{enumerate}[label=(\alph*), leftmargin=1\parindent]
\item F
\skipitems{1}
\item F
\skipitems{1}
\item T
\skipitems{1}
\item T


	\end{enumerate}
	
	
\ee

%15
\hspace{-0.25in} Section \ref{sec:bases_dimension}

\obe
\item 
\ba
\item $\left\{ \left[ \begin{array}{c} 1\\ 0 \\ 2\\ 1 \\ 3\\ 3 \end{array} \right], \left[ \begin{array}{c} 1\\ 0 \\ 0\\ 2 \\ 1\\ 3 \end{array} \right], \left[ \begin{array}{r} 0\\ 1 \\ 1\\ 1 \\ -1\\ 1 \end{array} \right] \right\}$, $\dim(\Col A) = 3$, $\dim(\Nul A) = 2$

\item $\left\{ \left[ \begin{array}{r} -3\\1\\0\\0\\0\end{array} \right], \left[ \begin{array}{r} 0\\0\\-2\\1\\0\end{array} \right] \right\}$

\ea
	
\item 
\ba
\item $\left\{ \left[ \begin{array}{r} 1 \\ -2 \\ 1 \end{array} \right] \right\}$, $\rank(A) = 1$ 

\item $\left\{ \left[ \begin{array}{r} -2\\1\\0\\0 \end{array} \right], \left[ \begin{array}{c} 1\\0\\1\\0 \end{array} \right], \left[ \begin{array}{c} 1\\0\\0\\1 \end{array} \right] \right\}$,  $\nullity(A) = 3$

\item $\rank(A) + \nullity(A) = 1+3 = 4$

\item $\left\{ \left[ \begin{array}{r} 1\\2\\-1\\-1 \end{array} \right] \right\}$, dimension $1$ 

	\ea

\item 
\ba
\item What are the solutions to $A \vx = \vzero$? 

\item Use part (a) and the Rank-Nullity Theorem.  

\ea

\item $\dim(W)$ can be $0$, $1$, $2$, $3$, or $4$ corresponding to $\{\vzero\}$, a line through the origin in $\R^4$, a plane containing the origin in $\R^4$,  a copy of $\R^3$ through the origin in $\R^4$, $\R^4$ 

\item  $\dim(\Col A) = 3$, $\dim(\Nul A) = 2$


\item ~
\ba
\item Not possible.

\item $\left[ \begin{array}{cc} 0&1\\0&0 \end{array} \right]$
\ea


	
\item 
\begin{enumerate}[label=(\alph*), leftmargin=1\parindent]
\item F
\skipitems{1}
\item F
\skipitems{1}
\item T
\skipitems{1}
\item T
\skipitems{1}
\item F
\skipitems{1}
\item T


\end{enumerate}



\oee

%16 Coordinate vectors and change of basis
\hspace{-0.25in} Section \ref{sec:coordinate_vectors}

\obe

\item $[\vb]_{\CB} = [-2 \ 3]^{\tr}$. 

\item $\left\{ [2 \ 1]^{\tr}, [1 \ 1]^{\tr}\right\}$, $\left\{ [1 \ 0]^{\tr}, [3 \ 3]^{\tr}\right\}$.  

\item $[1 \ 0 \ 2]^{\tr}$,  $[2 \ 1 \ 1]^{\tr}$ 
 
\item 
	\ba
	\item Row reduce an appropriate matrix. 
	 
	\item $[\vv_1]_{\CB} = [ -1 \ 0 \ 2]^{\tr}$, $[\vv_2]_{\CB} = [-1 \ 1 \ 2]^{\tr}$, $ [\vv_3]_{\CB} = [1 \ -1 \ 1]^{\tr}$ 

	\ea

\item Write $\vv$ and $\vw$ in terms of the basis vectors. 

\item 
\ba
\item Use the trigonometric identities $\cos(\theta+ \pi/2) = -\sin(\theta)$ and $\sin(\theta+ \pi/2) = \cos(\theta)$.

\item Find $\vx$ and then row reduce an appropriate matrix. 

\item $[\vy]_{\CB} = \left[ \renewcommand{\arraystretch}{1.3} \begin{array}{c} \sqrt{3}-\frac{3}{2} \\ 1+\frac{3\sqrt{3}}{2} \end{array} \right]$ 

\ea


\item $[ -1 \ 1 \ -2]^{\tr}$, $[1 \ 0 \ 1]^{\tr}$, $[1 \ 0 \ 2]^{\tr}$ 

\oee

\be
\item[14.]
\begin{enumerate}[label=(\alph*), leftmargin=1\parindent]
\item F
\skipitems{1}
\item F
\skipitems{1}
\item T
\skipitems{1}
\item T
\skipitems{1}
\item T
\skipitems{1}
\item T
\end{enumerate}

\ee


%17
\hspace{-0.25in} Section \ref{sec:determinants}

\obe
\item $\det(2A) = \\
(8a_{11})\left[(a_{22})(a_{33}) - (a_{23})(a_{32})\right]  - (8a_{12})\left[(a_{21})(a_{33}) - (a_{23})(a_{31})\right] + (8a_{13})\left[(a_{21})(a_{32}) - (a_{22})(a_{31})\right] = 8 \det(A).$


\item 
\ba
\item  $\det(A^2) = \det(AA) = \det(A) \det(A) = [\det(A)]^2$.

\item  $\det(A^k) = \det(AA^{k-1}) = \det(A) [\det(A)]^{k-1} = [\det(A)]^k$.

\item  Yes. 

\ea
	

\item $(\det(A))^2$

	
\item 
\begin{enumerate}[label=(\alph*), leftmargin=1\parindent]
\item F
\skipitems{1}
\item F
\skipitems{1}
\item T
\skipitems{1}
\item F
\skipitems{1}
\item T
\skipitems{1}
\item F


\end{enumerate} 

\oee

%18
\hspace{-0.25in} Section \ref{sec:characteristic_equation}

\obe
\item 
	\ba
	\item $\det(B) = -16$, eigenvalues $-2$ and $8$, $\det(B)$ is the product of the eigenvalues
	\item 
		\begin{enumerate}[label=\roman*.]
		\item Each $\lambda_i$ is a root of the characteristic polynomial. 
		\item Evaluate $p(\lambda)$ at $\lambda = 0$.
		\item Use the fact that $p(\lambda) = \det(A - \lambda I_n)$.
		\end{enumerate}
	
	\ea
	
	
\item Explain why $(A - \lambda I_n)^{\tr} = A^{\tr} - \lambda I_n$. 

\item Write $I_n$ as $B\lambda I_nB^{-1}$ and then expand $\det\left(BAB^{-1} - \lambda I_n \right)$.  

\oee

\be
\item[6.] 
\begin{enumerate}[label=(\alph*), leftmargin=1\parindent]
\item F
\skipitems{1}
\item T
\skipitems{1}
\item F
\skipitems{1}
\item T
\skipitems{1}
\item T
\end{enumerate}

\ee

%19
\hspace{-0.25in} Section \ref{sec:diagonalization}

\obe
\item
\ba
\item Not diagonalizable.

\item Diagonalizable by $P = \left[ \begin{array}{ccc} 1&2&1\\1&3&3\\1&3&4 \end{array} \right]$ 

\ea
 

\item $A = \left[ \begin{array}{cc} 1&2\\0&2 \end{array} \right]$, $P_1 = \left[ \begin{array}{cc} 1&2 \\ 0&1 \end{array} \right]$, we have $D_1 =  \left[ \begin{array}{cc} 1&0 \\ 0&2 \end{array} \right]$,  $P_2 = \left[ \begin{array}{cc} 2&1 \\ 1&0 \end{array} \right]$, $D_2 = \left[ \begin{array}{cc} 2&0 \\ 0&1 \end{array} \right]$

\item Yes

\item ~
\ba
\item Eigenvalues, 1; eigenvectors : $\Span \{[1 \ 0]^{\tr}$ 

\item  Diagonalizable

\ea


\item 
\ba
\item The $ii$ entry of $RS$ is $r_{i1}s_{1i} + r_{i2}s_{2i} + \cdots + r_{in}s_{ni}$. The $jj$ entry of $SR$ is $s_{j1}r_{1j} + s_{j2}r_{2j} + \cdots + s_{jn}r_{nj}$. Sum as $i$ and $j$ go from $1$ to $n$. 

\item 
	\begin{enumerate}[label=\roman*.]
	\item $\trace(D) = \trace\left(P^{-1}(AP)\right) = \trace\left((AP)P^{-1}\right) = \trace(A)$ 
	\item $\trace\left(A\right) = \trace(D) = \sum_{i=1}^n \lambda_i$
	\end{enumerate}
\ea

\item  
\ba
\item $e^A = \left[ \begin{array}{cc} e&e-1\\0&1 \end{array} \right]$

\item $e^B = I_2 + B = \left[ \begin{array}{cr} 1&-1 \\ 0&1 \end{array} \right]$.

\item $e^{A+B} =  \left[ \begin{array}{cc} e&0 \\ 0&1 \end{array} \right]$

\item No

\ea


\item 
\ba
\item 
\begin{align*}
e^A &= e^{PDP^{-1}} \\
	&= \sum_{k \geq 0} \frac{1}{k!} \left(PDP^{-1}\right)^k \\
	&=  \sum_{k \geq 0} \frac{1}{k!} PD^kP^{-1} \\
	&=  P\left(\sum_{k \geq 0} \frac{1}{k!} D^k\right) P^{-1} \\
	&= Pe^DP^{-1}.
\end{align*}

\item 
\begin{align*}
\det\left(e^A\right) &= \det\left(Pe^DP^{-1}\right) \\
	&=  \det\left(e^D\right) \\
	&= e^{\lambda_1}e^{\lambda^2} \cdots e^{\lambda_n} \\
	&= e^{\lambda_1+\lambda^2+ \cdots + \lambda_n} \\
	&= e^{\trace(A)}.
\end{align*}

\ea

\oee

\be
\item[15.] 
\begin{enumerate}[label=(\alph*), leftmargin=1\parindent]
\item T
\skipitems{1}
\item F
\skipitems{1}
\item T
\skipitems{1}
\item F
\skipitems{1}
\item T

\end{enumerate}

\ee



%20
\hspace{-0.25in} Section \ref{sec:approx_eigenvalues}

\obe
\item 
\ba
\item Eigenvalues: $3$ and $-1$; T$[1 \ -1]^{\tr}$ is an eigenvector for $A$ with eigenvalue $-1$ and $[1 \ 1]^{\tr}$ is an eigenvector for $A$ with eigenvalue $3$

\item As $k$ increases, the vectors $A^k \vx_0$ are approaching the vector $[1 \ 1]^{\tr}$, which is a dominant eigenvector of $A$.

\item The Rayleigh quotients $r_k = \frac{\vx_{k+1}\cdot \vx_k}{\vx_k \cdot \vx_k}$ approach the dominant eigenvalue $3$. 

\item Apply the power method to $B = (A - 0 I_2)^{-1} = A^{-1}$. As $k$ increases, the vectors $B^k \vx_0$ are approaching the vector $ \frac{1}{2}[1 \ -1]^{\tr}$, which is an eigenvector of $A$.

The Rayleigh quotients approach the other eigenvalue $-1$ of $A$. 

\ea

\item $[1 \ 1]^{\tr}$ is an eigenvector for $A$ with eigenvalue $1$, so the vectors $A^k \vx_0$ are all equal to $\vx_0$. We can adjust the seed to a non-eigenvector.

\item  $8$ is the dominant eigenvalue of $A$

\item 
\ba
\item Since $\R^n$ has dimension $n$, it follows that any set of $n+1$ vectors is linearly dependent.

\item Proceed down the list $c_{n-1}$, $c_{n-2}$, etc., until you reach a weight that is non-zero.

\item $\vzero = q(A) \vv = (A-\lambda I_n) Q(A)\vv$


\item 
	\begin{enumerate}[i.]
	\item $q(t) = 24t - 10t^2+t^3$
	\item $0$, $4$, and $6$ 
	\item For $t=0$, we have $Q(t) = (4-t)t-4)(t-6)$, and $[6 \ -6 \ 0]^{\tr}$ an eigenvector for $A$ with eigenvalue $0$.
	
	For $t=4$ we have $Q(t) = t(t-6)$, and $[24 \ 24 \ 0]^{\tr}$ is an eigenvector for $A$ with eigenvalue $4$. 
	
	For $t=6$ we have $Q(t) = t(t-4)$, and $[0 \ -6 \ -12]^{\tr}$ is an eigenvector for $A$ with eigenvalue $6$. 
	\end{enumerate}

\ea

\item If $\beta$ is an eigenvalue of $B$ with eigenvector $\vx$, then $\frac{1}{\beta} + \alpha$ is an eigenvalue of $A$ with eigenvector $\vx$. 

\oee

\be
\item[10.]
\begin{enumerate}[label=(\alph*), leftmargin=1\parindent]
\item F
\skipitems{1}
\item T
 

\end{enumerate}

\ee

%21
\hspace{-0.25in} Section \ref{sec:complex_eigenvalues}

\obe
\item 
\ba
\item Eigenvalues:$\lambda_1 = 2+2\sqrt{2}i$ and $\lambda_2 = 2-2\sqrt{2}i$; Eigenvectors:  $[-\sqrt{2}i \ 1]^{\tr}$ and  $[\sqrt{2}i \ 1]^{\tr}$

\item Eigenvalues: $\lambda_1 = 2+i$ and $\lambda_2 = 2-i$; Eigenvectors:  $[-(1+i) \ 1]^{\tr}$,  $[-(1-i) \ 1]^{\tr}$

\item Eigenvalues: $\lambda_1 = -1+2i$ and $\lambda_2 = -1-2i$; Eigenvectors:   $[(1+i) \ 2]^{\tr}$ and $[(1-i) \ 2]^{\tr}$

\ea

\item Just the rotation matrices

\item $\left[ \begin{array}{cr} 1&-2 \\ 2&1 \end{array} \right]$

\item 
\ba
\item Characteristic polynomial $\lambda^2+a_1\lambda + a_0$; $\left[ \begin{array}{cr} 0&-2\\1&2 \end{array} \right]$

\item $\det(C - \lambda I_3) = -\lambda^3 -a_2 \lambda^2 - a_1 \lambda - a_0$


\ea

\oee

\be
\item[8.]
\begin{enumerate}[label=(\alph*), leftmargin=1\parindent]
\item T
\skipitems{1}
\item F
\skipitems{1}
\item T
\skipitems{1}
\end{enumerate}



\ee

%22
\hspace{-0.25in} Section \ref{sec:det_properties}

\obe
\item $\det(rA) = r^{n} \det(A)$

\item 
\ba
\item  $A_1=  \left[ \begin{array}{rrrr} 1&1&1&1 \\ -1&4&-1&-1\\ -1&-1&4&-1\\ -1&-1&-1&4 \end{array} \right]$

\item $B=  \left[ \begin{array}{cccc} 1&1&1&1 \\ 0&5&0&0\\ 0&0&5&0\\ 0&0&0&5 \end{array} \right]$

\item $125 = \det(B) = \det(A_1) = \det(A)$

\item $\det(A) =  (n+1)^{n-1}$

\ea

\item $A^{-1} = \frac{1}{\det(A)} \adj(A)$ \\
$ = \left[ \begin{array}{rcr} -1&0&1 \\ 0&1&0 \\ 2&0&-1 \end{array} \right]$

\item 
\ba
\item $6$ 

\item $-16$

\ea

\item $|c-7|$, volume is $0$ when the parallelelepiped is two-dimensional. 

\oee

\be
\item[10.]
\begin{enumerate}[label=(\alph*), leftmargin=1\parindent]
\item T
\skipitems{1}
\item T
\skipitems{1}
\item F
\skipitems{1}
\item T

\end{enumerate}

\ee


%23
\hspace{-0.25in} Section \ref{sec:dot_product}
\obe
\item 
\ba
\item The angle between $\vu$ and $\vv$ is $\frac{\pi}{2}$. The distance between $\vu$ and $\vv$ is $\sqrt{10}$. The orthogonal projection of $\vu$ onto $\vv$ is $\vzero$.

\item The angle between $\vu$ and $\vv$ is $0$. The distance between $\vu$ and $\vv$ is $\sqrt{2}$. $\proj_{\vv} \vu = \vu$. 

\item The angle between $\vu$ and $\vv$ is approximately $104.96^{\circ}$. The distance between $\vu$ and $\vv$ is $\sqrt{17}$. $\proj_{\vv} \vu = \frac{\vu \cdot \vv}{||\vv||^2} \vv = -\frac{1}{10}[1 \ 3]^{\tr}$.

\item The angle between $\vu$ and $\vv$ is $\frac{\pi}{2}$. The orthogonal projection of $\vu$ onto $\vv$ is $\vzero$. The distance between $\vu$ and $\vv$ is $\sqrt{11}$.

\item The angle between $\vu$ and $\vv$ is approximately $54.74^{\circ}$.
The distance between $\vu$ and $\vv$ is $\sqrt{2}$. $\proj_{\vv} \vu = \frac{\vu \cdot \vv}{||\vv||^2} \vv = \frac{1}{3}[1 \ 1 \ 1]^{\tr}$. 

\ea

\item $h = 6 \pm 4\sqrt{3}$ 

\item Write the dot product as a matrix-vector product. 

\item \label{ex:Pyth_Thm} 
\ba	
\item Write the norm as a dot product. 

\item From part (a), what can we say about $2(\vu \cdot \vv)$?  

\ea
	
\item Consider cases of $\vu \cdot \vv \leq 0$ separately. Then write the norm as a dot product. 

\item Use the definition of $W^\perp$. 

\item If $\vw \in W_2^\perp$ and $\vv \in W_1$, in what other set is $\vv$? 

\oee

\be
\item[15.]
\begin{enumerate}[label=(\alph*), leftmargin=1\parindent]
	\item F
\skipitems{1}
	\item F 
\skipitems{1}
	\item F 
\skipitems{1}	
	\item F
\skipitems{1}	
	 \item T
\skipitems{1}
	\item T
\skipitems{1}
	\item F 
\skipitems{1}	
	\item T
\skipitems{1}	 
	\item T
	\end{enumerate}
	
\ee



%24
\hspace{-0.25in} Section \ref{sec:orthogonal_basis}
\obe
\item $\left\{ [0 \ 1 \ 0]^{\tr}, \left[ \frac{3}{4} \ 0 \ 1\right]^{\tr} \right\}$

\item \label{problem:orthog_decomp} 
\ba
\item Where are $\vw_1$, $\vw_2$, $\ldots$, $\vw_k$? 

\item Take the dot product of $\vz$ with $\vw_i$. 

\item Use parts (a) and (b).

\item Collect terms in $W$ and in $W^{\perp}$. 

\ea

\item Consider $(P \ve_i) \cdot (P \ve_j)$ where $\ve_t$ is the $t$th standard basis vector for $\R^n$.

\item 
\ba
\item Think polar cordinates. 

\item What properties do the columns of an orthogonal matrix have?   

\item What happens if $\alpha = \theta+\frac{\pi}{2}$ and if  $\alpha = \theta-\frac{\pi}{2}$?

\ea

\oee

\be
\item[9.]
\begin{enumerate}[label=(\alph*), leftmargin=1\parindent]	
	\item F
\skipitems{1}
	\item T
\skipitems{1}	
	\item F  
\skipitems{1}
	\item T
\skipitems{1}	
	\item T
\skipitems{1}	
	\end{enumerate}


\ee


%25
\hspace{-0.25in} Section \ref{sec:gram_schmidt}

\obe
\item 
\ba
\item $\proj_W \vv =  \frac{1}{2} [1 \ 0 \ 1]^{\tr}$, $\proj_{\perp W} \vv =  \frac{1}{2}[1 \ 0 \ -1]^{\tr}$

\item $ \frac{1}{2} [1 \ 0 \ 1]^{\tr}$,  $\sqrt{\frac{1}{2}}$

\ea

\item  $y = \frac{3}{2}x + \frac{2}{3}$. 
	
\item Let $\CB = \{\vw_1, \vw_2, \ldots, \vw_m\}$ be an orthogonal basis for a subspace $W$ of $\R^n$, and let $\vy$ be a vector in $\R^n$ that is orthogonal to $\vw_k$ for every $k$. Show that $\vy$ is orthogonal to every vector in $W$. 

\item 
\ba
\item Orthogonal basis: $\{[1 \ 1 \ 1]^{\tr},[3  -3 \ 0]^{\tr}\}$ 

\item Orthogonal basis:  
\[\{[1 \ 0 \ 2]^{\tr}, \frac{1}{7}[-9 \ 1 \ 3]^{\tr}, \frac{1}{2}[1 \ 2 \ -2]^{\tr}\}\] 

\item Orthogonal basis vectors: 
\begin{tabular}{l}
$[1 \ 0 \ 1 \ 0 \ 1 \ 0 \ 1]^{\tr}$ \\ 
$[-2 \ 2 \ 2 \ 0 \ 0 \ 0 \ 0]^{\tr}$ \\
$[0 \ -1 \ 1 \ -1 \ 0 \ 0 \ -1]^{\tr}$ \\
$[1/3 \ 1/15 \ 4/15 \ 2/5 \ 0 \ 1 \ -3/5]^{\tr}$
\end{tabular}
%\[\left\{\left[ \begin{array}{c} 1 \\ 0 \\ 1 \\ 0 \\ 1 \\ 0 \\ 1\end{array} \right],  \left[ \begin{array}{r} -2 \\ 2 \\ 2 \\ 0 \\ 0 \\ 0 \\ 0\end{array} \right],  \left[ \begin{array}{r} 0 \\ -1 \\ 1 \\ -1 \\ 0 \\ 0 \\ -1\end{array} \right], \left[ \begin{array}{r} 1/3 \\ 1/15 \\ 4/15 \\ 2/5 \\ 0 \\ 1 \\ -3/5\end{array} \right] \right\}\]

\ea
 	
\item Not possible.

\item $\left\{  \frac{1}{\sqrt{2}} \left[ \begin{array}{c} 1 \\ 1  \\ 0 \end{array} \right],  \frac{1}{\sqrt{38}} \left[ \begin{array}{r } -1\\ 1 \\ 6  \end{array} \right], \frac{1}{\sqrt{19}}\left[ \begin{array}{r}[3 \\ -3 \\ 1  \end{array} \right] \right\}$

\item \label{ex:6_e_upper_triangular_props} 
\ba
\item Write the $ij$th entry of $RS$ in terms of the entries in $R$ and $S$.  

\item Use $RR^{-1} = I$ to write a system of equations using the columns of $R^{-1}$. 
\ea

\oee

\be
\item[13.]
\begin{enumerate}[label=(\alph*), leftmargin=1\parindent]	
	\item F
	\skipitems{1}
	\item T 
	\skipitems{1}
	\item T
	\skipitems{1}
	\item T
	\end{enumerate}
	
	
\ee


%26
\hspace{-0.25in} Section \ref{sec:least_squares}

\obe
\item 
\ba
\item $f(x) \approx 35.5273 - 0.0527x$

\item Approximately $32.10.$ 

\ea


\item $f(x) \approx 25.124 +  3.297x$

\item \label{ex:6_f_not_li} 
\ba
\item $\left[ \begin{array}{cc} 1&2 \\ 1&2 \\ 1&2 \end{array} \right] \left[ \begin{array}{c} a_0 \\ a_1 \end{array} \right] = \left[ \begin{array}{c} 1\\2\\3 \end{array} \right]$

\item $A^{\tr}A$ is not invertible

\item Infinitely many.

\item $x=2$

\ea

\item \label{ex:6_f_LS_solutions} 
\ba
\item The augmented column of $[M^{\tr}M \ | \ M^{\tr}\vy]$ cannot be a pivot column. 

\item Use Exercise \ref{ex:6_f_ranks}. 

\item Use the definition of the matrix product.

\item Use Exercise \ref{ex:6_f_ranks}.  

\item Combine parts (b) and (d).  

\ea

\oee

\be
\item[9.]
\begin{enumerate}[label=(\alph*), leftmargin=1\parindent]	
	\item T
	\skipitems{1}
	\item F 
	\skipitems{1}
	\item T
	\end{enumerate}
	
	
\ee

%27
\hspace{-0.25in} Section \ref{sec:orthogonal_diagonalization}
\obe
\item 
\ba
\item $\frac{1}{\sqrt{5}} \left[ \begin{array}{rc} -2&2\\1&1 \end{array} \right]$ 

\item $\left[\renewcommand{\arraystretch}{1.3} \begin{array}{crr} \frac{1}{\sqrt{3}}&-\frac{2}{\sqrt{6}}&0 \\ \frac{1}{\sqrt{3}}&\frac{1}{\sqrt{6}}&-\frac{1}{\sqrt{2}}\\ \frac{1}{\sqrt{3}}&\frac{1}{\sqrt{6}}&\frac{1}{\sqrt{2}} \end{array} \right]$

\item $A$ is not orthogonally diagonalizable.
 
\ea

\item $\frac{1}{2} \left[ \begin{array}{cccc} 9&3&0&0 \\ 3&9&0&0 \\ 0&0&4&0 \\ 0&0&0&4 \end{array} \right]$

 
\item \label{ex:7_a_spectral_decomposition} 
\ba
\item Determine $P_i^{\tr}$ 

\item  Show that every column of $P_i$ is a scalar multiple of $\vu_i$.  

\item  Use the orthonormal basis to simplify $P_i^2$. 

\item Use the orthonormal basis to simplify $P_iP_j$. 

\item Use the orthonormal basis to simplify $P_i\vu_i$. 

\item Use the orthonormal basis to simplify $P_i\vu_j$. 

\item Recall that $\proj_{W_i} \vv = \frac{\vv \cdot \vu_i}{\vu_i \cdot \vu_i} \vu_i$. 

\ea

	
\item 
\ba
\item A basis for the eigenspace of $A$ corresponding to the eigenvalue $-1$ is $\{[0 \ 0 \ -2 \ 1]^{\tr}, [-2 \ 1 \ 0 \ 0]^{\tr}\}$ and a basis for the eigenspace of $A$ corresponding to the eigenvalue 4 is $\{[0 \ 0 \ 1 \ 2]^{\tr}, [1 \ 2 \ 0 \ 0]^{\tr}\}$. $P_1 = \frac{1}{5} \left[ \begin{array}{ccrr} 0&0&0&0 \\ 0&0&0&0 \\ 0&0&4&-2 \\ 0&0&-2&1 \end{array} \right]$,  
$P_2 = \frac{1}{5} \left[ \begin{array}{rrcc} 4&-2&0&0 \\ -2&1&0&0 \\ 0&0&0&0 \\ 0&0&0&0 \end{array} \right]$,  
$P_3 = \frac{1}{5} \left[ \begin{array}{cccc} 0&0&0&0 \\ 0&0&0&0 \\ 0&0&1&2 \\ 0&0&2&41 \end{array} \right]$,
$P_4 = \frac{1}{5} \left[ \begin{array}{cccc} 1&2&0&0 \\ 2&4&0&0 \\ 0&0&0&0 \\ 0&0&0&0 \end{array} \right]$, 
$\mu_1 = -1$, $\mu_2 = 4$,  $Q_1  = \frac{1}{5} \left[ \begin{array}{rrrr} 4&-2&0&0 \\ -2&1&0&0 \\ 0&0&4&-2 \\ 0&0&-2&1 \end{array} \right]$, and 
$Q_2 =  \frac{1}{5} \left[ \begin{array}{cccc} 1&2&0&0 \\ 2&4&0&0 \\ 0&0&1&2 \\ 0&0&2&4 \end{array} \right]$.


\item 
		\begin{enumerate}[i.]
		\item Collect matrices with the same eigenvalues.  
		
		\item Use the fact that each $P_i$ is a symmetric matrix. 
		
		\item Use Theorem 31.8. 
		
		\item Use Theorem 31.8.  
		
		\item Explain why $\{\vu_{1_j}$, $\vu_{2_j}$, $\ldots$, $\vu_{m_j}\}$ is a orthonormal basis for $E_{\mu_j}$. 
		
		\end{enumerate}

\item The rank of $Q_j$ is $m_j$.  

\ea

\oee

\be
\item[8.]
\begin{enumerate}[label=(\alph*), leftmargin=1\parindent]	
	\item T
	\skipitems{1}
	\item F 
	\skipitems{1}
	\item T
\skipitems{1}
	\item T
\skipitems{1}
	\item T
\skipitems{1}
	\item F 

	\end{enumerate}
	
	
\ee

%28
\hspace{-0.25in} Section \ref{sec:principal_axis_theorem} 
\obe
\item 
\ba
\item $\left[ \begin{array}{rr} 1&-1\\-1&4 \end{array} \right]$

\item $\left[ \begin{array}{ccc} 10&0&2\\0&0&1 \\ 2&1&1 \end{array} \right]$

\item $\left[\renewcommand{\arraystretch}{1.3}  \begin{array}{rccr} 0&1&1&-\frac{1}{2}\\1&5&0&0 \\ 1&0&0&2 \\ -\frac{1}{2}&0&2&8 \end{array} \right]$

\ea
	

\item \label{ex:7_b_constrained_optimization} 
 \ba
 \item 
 		\begin{enumerate}[i.]
		\item Let $P$ be a matrix that orthogonally diagonalizes $A$, with $P^{\tr}AP = D$. Use this to calculate $Q(\vx)$. 

 		\item  Substitute in part i. 
				
		 \item Make an argument similar to part ii. 
		 
		\end{enumerate}
 
 \item  The maximum value of $Q(\vx)$ on the unit circle is 1 and it occurs at the input $\frac{1}{\sqrt{2}}[1 \ -1]$. 
 
 \ea
 

\item \label{ex:7_b_QF_characterization} 
\ba
\item $1 = \lambda_1 y_1^2 + \lambda_2 y_2^2$. 

\item An ellipse.

\item A hyperbola.

\item Two lines. 

\ea

\item \label{ex:7_b_unique_A}  Use the previous exercise to compare $Q_A(\ve_i)$ and $Q_B(\ve_i)$, then compare $Q_A(\ve_i+\ve_j)$ to $Q_B(\ve_i + \ve_j)$ for $i \neq j$.

\item $I_n$

\item 
\ba
\item Expand $|| \vu - \vv ||^2$ using the inner product.

\item Expand $|| \vu - \vv ||^2$ using the inner product.

\ea

\item Expand $|| \vu - \vv ||^2$ using the inner product.

\oee

\be
\item[14.]
\begin{enumerate}[label=(\alph*), leftmargin=1\parindent]	
	\item F 
\skipitems{1}	
	\item T
\skipitems{1}	
	\item T
\skipitems{1}	
	\item F 
\skipitems{1}	
\item T
\skipitems{1}
\item T
\skipitems{1}
\item T
\skipitems{1}
\item T

	 \end{enumerate}
	 
\ee

%29
\hspace{-0.25in} Section \ref{sec:SVD} 
\obe
\item 
\ba
\item $U = \left[ \begin{array}{cc} 1&0\\0&1 \end{array} \right]$,  $\Sigma = \left[ \begin{array}{cc} \sqrt{2}&0\\0&0 \end{array} \right]$, $V = \frac{1}{\sqrt{2}}\left[ \begin{array}{cr} 1&-1\\1&1 \end{array} \right]$.

\item $U = \frac{1}{\sqrt{2}}\left[ \begin{array}{c} 1\\0\\1\end{array} \right]$, $\Sigma = [\sqrt{2}]$, $V = [1]$.

\item $U = \frac{1}{\sqrt{2}}\left[ \begin{array}{cr} 1&-1\\1&1 \end{array} \right]$, 
$\Sigma = \left[ \begin{array}{ccc} \sqrt{3}&0&0\\0&1&0 \end{array} \right]$, 
$V = \left[ \renewcommand{\arraystretch}{1.4} \begin{array}{crr} \frac{2}{\sqrt{6}}&0&-\frac{1}{\sqrt{3}}\\ \frac{1}{\sqrt{6}}&-\frac{1}{\sqrt{2}}&\frac{1}{\sqrt{3}} \\ \frac{1}{\sqrt{6}}&\frac{1}{\sqrt{2}} &\frac{1}{\sqrt{3}}\end{array} \right]$.

\item $U = \left[ \arraycolsep=0.75pt \renewcommand{\arraystretch}{1.4} \begin{array}{crrr} \frac{1}{\sqrt{10}}&\frac{1}{\sqrt{10}}&\frac{2}{\sqrt{5}}&0 \\ \frac{1}{\sqrt{10}}&-\frac{1}{\sqrt{10}}&0&-\frac{2}{\sqrt{5}} \\ \frac{2}{\sqrt{10}}&-\frac{2}{\sqrt{10}}&0&\frac{1}{\sqrt{5}} \\ \frac{2}{\sqrt{10}}&\frac{2}{\sqrt{10}}&-\frac{2}{\sqrt{5}}&0  \end{array} \right]$, 
$\Sigma = \left[ \begin{array}{cc} 5&0 \\ 0&\sqrt{5} \\ 0&0\\0&0 \end{array} \right]$, 
$V= \frac{1}{\sqrt{2}} \left[ \begin{array}{cr} 1&-1 \\ 1&1 \end{array} \right]$.

\item $U = \left[ \renewcommand{\arraystretch}{1.4} \begin{array}{ccr} 0&1&0 \\ \frac{1}{\sqrt{2}}&0&-\frac{1}{\sqrt{2}} \\ \frac{1}{\sqrt{2}}&0&\frac{1}{\sqrt{2}}  \end{array} \right]$,  $\Sigma = \left[ \begin{array}{cccc} 3&0&0&0 \\ 0&2&0&0 \\ 0&0&1&0 \end{array} \right]$, $V =  \left[ \renewcommand{\arraystretch}{1.4} \begin{array}{ccrc} 0&1&0&0 \\ \frac{1}{\sqrt{2}}&0&-\frac{1}{\sqrt{2}}&0 \\ \frac{1}{\sqrt{2}}&0&\frac{1}{\sqrt{2}}&0 \\ 0&0&0&1  \end{array} \right]$.

\ea


\item 
\ba
\item $\sqrt{28}$

\item $U = \left[ \renewcommand{\arraystretch}{1.3} \begin{array}{crr} \frac{1}{\sqrt{14}}&\frac{2}{\sqrt{5}}&\frac{3}{\sqrt{70}} \\ \frac{2}{\sqrt{14}}&-\frac{1}{\sqrt{5}}&\frac{6}{\sqrt{70}} \\ \frac{3}{\sqrt{14}}&0&-\frac{5}{\sqrt{70}} \end{array} \right]$,  $\Sigma = \left[ \begin{array}{cc} \sqrt{28}&0 \\ 0&0 \\ 0&0 \end{array} \right]$, $V = \frac{1}{\sqrt{2}}\left[ \begin{array}{cr} 1&-1 \\ 1&1 \end{array} \right]$.


\item
	\begin{enumerate}[i.]	
	\item $\left\{\frac{1}{\sqrt{2}}[-1 \ 1]^{\tr}\right\}$.
	\item $\left\{\frac{1}{\sqrt{14}}[1 \ 2 \ 3]^{\tr}\right\}$. 
	\item $\left\{ \frac{1}{\sqrt{2}}[1 \ 1]^{\tr}\right\}$. 

		\end{enumerate}

\ea 


\item Find the transpose of an SVD for $A$. 

\item  $||A|| = \lambda_1$ 

\item \label{ex:7_c_rank_1} Mimic Exercise 5 in Section 27.

\oee

\be
\item[11.]
\begin{enumerate}[label=(\alph*), leftmargin=1\parindent]	
	\item F 
\skipitems{1}	
	\item T
\skipitems{1}	
	\item F 
\skipitems{1}	
	\item F 	
\end{enumerate}
	
\ee

%30
\hspace{-0.25in} Section \ref{sec:pseudoinverses}
\obe
\item 
\ba
\item $60$, $15$, and $6$

\item 
\begin{align*}
&A =  4\left([3 \ 0 \ 4]^{\tr}\right)\left([2 \ 1 \ 1]\right) \\
	&- \left([4 \ 0 \ -3]^{\tr}\right)\left([1 \ 2 \ -2]\right) \\
	&+ 2\left(\frac{1}{5}[0 \ 5 \ 0]^{\tr}\right)\left([-2 \ 2 \ 1]\right).
\end{align*}

\item $\left[ \begin{array}{ccc} 24&12&24 \\ 0&0&0 \\ 32&16&32 \end{array} \right]$, $\frac{261}{3861} \approx 0.068$.

\item $\left[ \begin{array}{ccc} 20&4&32 \\ 0&0&0 \\ 35&22&26 \end{array} \right]$,  $\frac{36}{3861} \approx 0.009$.

\ea

\item 
\ba
\item The ball goes up and then it comes down. 

\item $\frac{266}{97} + \frac{12213}{970}x -\frac{1423}{9700}x^2$.

\item Approximately $23.9$ and $61.9$ degrees. 

\ea

\item 
	\ba
	\item The eigenvalues for $A$ are 2, 1, and 0, the eigenvalues for $A^{\tr}A$ are 4, 1, and 0 with the same corresponding eigenvectors. 
	
	\item $\left[ \begin{array}{rcc} 0&1&0 \\ -\frac{1}{\sqrt{2}}&0&\frac{1}{\sqrt{2}}\\ \frac{1}{\sqrt{2}}&0&\frac{1}{\sqrt{2}} \end{array} \right]$
	
	\item $B =  \left[ \renewcommand{\arraystretch}{1.4} \begin{array}{crr} 1&0&0 \\ 0&\frac{1}{\sqrt{2}}&-\frac{1}{\sqrt{2}} \\0&-\frac{1}{\sqrt{2}}&\frac{1}{\sqrt{2}} \end{array} \right]$
	
	\item $A$ was symmetric to write a singular value decomposition for $A$ in the form $V \Sigma V^{\tr}$, $A$ is positive definite
	
	\item $\left[ \renewcommand{\arraystretch}{1.4} \begin{array}{crr} 1&0&0 \\ 0&\frac{\sqrt[3]{2}}{2}&-\frac{\sqrt[3]{2}}{2} \\0&-\frac{\sqrt[3]{2}}{2}&\frac{\sqrt[3]{2}}{2} \end{array} \right]$
	
	\ea


\item 
\ba
\item Use the fact that $X= XAX$.

\item Use the fact that $Y= YAY$.

\item Compare the results of (a) and (b).

\ea

\item Use the Rank-Nullity Theorem.
   

\oee

\be
\item[11.]
\begin{enumerate}[label=(\alph*), leftmargin=1\parindent]	
	\item F 
	\skipitems{1}
	\item F 
	\skipitems{1}
	\item F 	
	\end{enumerate}
	
\ee

%31
\hspace{-0.25in} Section \ref{sec:vector_spaces}

\obe
\item 
\ba
\item Use the fact that $\vzero_1+\vv = \vv$ for any vector $\vv$ in our vector space. 

\item Same reasoning as in part (a). 

\item Use the transitive property of equality.  

\ea

\item Use the fact that $-1 + 1 = 0$. 
	
\item The intersection $W_1 \cap W_2$ is a subspace of $V$, but the union $W_1 \cup W_2$ is not in general a subspace of $V$. 
 
\item The space $W$ is the span of  $\left[ \begin{array}{cc} 1 & 1 \\ 0 & 1 \end{array} \right]$,  $\left[ \begin{array}{cc} 1 & 0 \\ 3& 1 \end{array} \right]$, and $\left[ \begin{array}{cr} 0 & -2 \\ 1 & 1 \end{array} \right]$. 

\item Mimic the proof of Theorem 12.5. 

\item 
\ba
\item  Closure is by definition, the sequence $\{0\}$ is the additive identity, the sequence $\{-x_n\}$ is the additive inverse of the sequence $\{x_n\}$. The other properties follow from the definitions of addition and multiplication by scalars.   

\item  The answer is no. 

\item  The answer is yes. 

\item  The answer is no. 

\item  The answer is yes. 

\item  To show that $\ell^2$ is closed under addition, expand the square. 

\ea
	
\oee

\be	
\item[13.] 
\begin{enumerate}[label=(\alph*), leftmargin=1\parindent]
\item T 
\skipitems{1}
\item T
\skipitems{1}
\item T
\skipitems{1}
\item T
\skipitems{1}
\item T
\end{enumerate}	
	
\ee


%32
\hspace{-0.25in} Section \ref{sec:bases}

\obe
\item 
\ba
\item This set is a basis for $\R^3$. 

\item The set is linearly independent in $\pol_3$ but does not span $\pol_3$. 

\item The set is a basis for $\pol_3$. 

\item The set is linearly independent in $M_{3 \times 2}$ but does not span $M_{3 \times 2}$.  

\ea 
	
\item The set $\{M_{ij}\}$ where $M_{ij}$ is the matrix with a $1$ in the $ij$th position and zeros everywhere else is a basis for $\M_{2 \times 2}$, as is the set $\{M'_{ij}\}$, where $M'_{ij}$ is the matrix with a $-1$ in the $ij$th position and zeros everywhere else. 

\item  The set is a basis for $V$. 

\item It is not possible. 

\item Mimic the proof of Theorem 6.2.

\item Mimic the proof of Theorem 6.4.

\oee

\be
\item[13.]
\begin{enumerate}[label=(\alph*), leftmargin=1\parindent]
\item F
\skipitems{1}
\item T
\skipitems{1}
\item T
\skipitems{1}
\item F 
\skipitems{1}
\item F 
\skipitems{1}
\item T

\ee


\ee

%33
\hspace{-0.25in} Section \ref{sec:dimension}

\obe
\item The set $\{ 1+t^2, 2+t+2t^2+t^3, 1+t+t^3\}$ is a basis for $W$ and $\dim(W) = 3$. 

\item 
\ba
\item No. 

\item The set $S$ is linearly dependent.

\item The set $\{A,B,E\}$ forms a basis for $\Span \ S$ and $C = 2A - 3B$, $D = 3A-2B$.

\item Let $F=\left[ \begin{array}{ccc} 0&0&0\\1&0&0 \end{array} \right]$,  $G=\left[ \begin{array}{ccc} 0&0&0\\0&1&0 \end{array} \right]$, and $H=\left[ \begin{array}{ccc} 0&0&0\\0&0&1 \end{array} \right]$. The set $\{A,B,E,F,G,H\}$ is a basis for $\M_{2 \times 3}$.


\ea

	
\item The set 
\[\CB = \left\{\left[ \begin{array}{cr} 1&0\\0&-1 \end{array} \right], \left[ \begin{array}{cc} 0&1\\0&1 \end{array} \right], \left[ \begin{array}{cc} 0&0\\1&0 \end{array} \right]\right\}\] is a basis for $W$. It follows that $\dim(W) = 3$. 


\item The set  $\CB = \{-1+t, -1+t^2\}$ is a basis for $W$ and $\dim(W) = 2$. 

\item Consider dimensions.  

\item 
\ba
\item Consider the row, column, and diagonal sums.

\item Set up a system of equations. The dimension is 3.

\ea

\oee

\be
\item[12.]

\item 
\begin{enumerate}[label=(\alph*), leftmargin=1\parindent]
\item T
\skipitems{1}
\item T
\skipitems{1}
\item F  
\skipitems{1}
\item T
\skipitems{1}
\item F 
\skipitems{1}
\item T

\end{enumerate}


\ee

%34
\hspace{-0.25in} Section \ref{sec:coordinate_vectors_vector_spaces}

\obe
\item $[\vb]_{\CB} = \left[ \begin{array}{r} -2\\3 \end{array} \right]$. 

\item Two such bases are $\{[1 \ 0]^{\tr}, [3 \ 3]^{\tr}\}$ and $\{[1 \ 1]^{\tr}, [3 \ 1]^{\tr}\}$. 

\item $\CB = \{ [1 \ 0 \ 2]^{\tr},  [2 \ 1 \ 1]^{\tr}\}$.

\item 
\ba
\item Show that $\CB$ is linearly independent. 

\item 
	\begin{enumerate}[i.]
	\item $[p_1(t)]_{\CB} = [1 \ 0 \ 1]^{\tr}$, $[p_2(t)]_{\CB} = [1 \ 1 \ 1]^{\tr}$, and $[p_3(t)]_{\CB} = [2 \ -1 \ -1]^{\tr}$.

	\item Row reduce the matrix $\left[ \begin{array}{ccr} 1&1&2 \\ 0&1&-1 \\ 2&2&1 \end{array} \right]$.  
	\end{enumerate}
\ea

\item 
\ba
\item Show that $\CB$ is linearly independent.  

\item $[A]_{\CB} = [0 \ 0 \ 0 \ 1]^{\tr}$, $[B]_{\CB} = [0 \ 1 \ 1 \ 0]^{\tr}$, $[C]_{\CB} = [1 \ 1 \ 1 \ 0]^{\tr}$, and $[D]_{\CB} = [1 \ -1 \ -1 \ 1]^{\tr}$.

\item Row reduce $[ [A]_{\CB} \ [B]_{\CB} \ [C]_{\CB} \ [D]_{\CB}]$.  

\ea
	
\item Use the fact that $c\vu = c(u_1\vv_1 + u_2\vv_2 + \cdots + u_n\vv_n) = (cu_1)\vv_1 + (cu_2)\vv_2 + \cdots + (cu_n) \vv_n$. 

\item 
\ba
\item Write $\vx$ as a linear combination of basis vectors. 

\item Apply $T$ to an appropriate vector and use the fact that $T$ is one-to-one. 	

\ea

\item Find a vector $\vy$ in $V$ such that $\vx = [\vy]_{\CB}$.

\oee

\be
\item[17.]
\begin{enumerate}[label=(\alph*), leftmargin=1\parindent]
\item F 
\skipitems{1}
\item F
\skipitems{1}
\item T
\skipitems{1}
\item T
\skipitems{1}
\item T
\end{enumerate}


\ee

%35
\hspace{-0.25in} Section \ref{sec:inner_products}
\obe

\item \label{ex:6_c_Cab} Use properties of continuous functions. 

\item 
\ba
\item $2x^2+3y^2 = 1$ 

\item An ellipse centered at the origin with major axis the segment from $\left(0, -\frac{1}{\sqrt{3}}\right)$ to $\left(0, \frac{1}{\sqrt{3}}\right)$ and minor axis the segment from $\left( -\frac{1}{\sqrt{2}}, 0\right)$ to $\left( \frac{1}{\sqrt{2}}, 0\right)$. 

\ea


\item \label{ex:6_c_ip_2} 
\ba
\item Verify the inner product properties. Why is the assumption that the $a_i$ are positive necessary? 

\item  $A = \left[ \begin{array}{ccccc} a_1&0&0&\cdots &0 \\ 0&a_2&0&\cdots &0 \\ &&&\ddots& \\ 0&0&0&\cdots &a_n \end{array} \right]$. 

\ea

 
\item 
\ba
\item No.

\item If $A$ is a diagonal matrix with positive diagonal entries. 

\ea


\item 
\ba 
\item Evaluate each side of the inequality. 

\item Write $||\vw||^2$ as an inner product and expand. 

\ea

	
\item 
\ba
\item Expand the inner product. 

\item Expand the inner product. 

\item Convert $A$ and $B$ to vectors in $\R^{n^2}$ whose entries are the entries in the first row followed by the entries in the second row and so on.   

\ea

\item  
\ba
\item Compute the inner product of the vectors. 

\item Try to write $\vv$ in terms of the basis vectors for $W$. 

\item $[1 \ 2 \ 1]^{\tr}$. Approximately 1.41.

\ea


\item 
\ba
\item Use the fact that $\vzero = \vzero + \vzero$.  

\item Use the fact that $\langle \vu, \vv \rangle = \langle \vv, \vu \rangle$.  

\item Same hint as part (b). 

\item Use the fact that $\vu - \vv = \vu + (-\vv)$. 

\ea

\item Mimic Theorem 24.3.

\oee


\be
\item[19.]
\begin{enumerate}[label=(\alph*), leftmargin=1\parindent]	
	\item F 
	\skipitems{1}
	\item F
	\skipitems{1}	
	\item T 
	\skipitems{1}
	\item T   
	\skipitems{1}
  	\item F
	\skipitems{1}
  	\item T
	\skipitems{1}
  	\item T
	\skipitems{1}
  	\item T
	\end{enumerate}    

\ee


%36
\hspace{-0.25in} Section \ref{sec:gram_schmidt_ips}
\obe

\item  
\ba
\item $\proj_W h(t) \approx -0.386t^2 -0.721t + 2.06$

\item$ 2-2t^2$

\item The least squares polynomial is an overall better fit. 

\ea
	
\item 
\ba
\item Evaluate an appropriate linear combination at several well chosen values to set up a linear system. 

\item $\left\{1, \cos(t), \sin(t) - \frac{2}{\pi}\right\}$

\ea

\item 
\ba
\item Evaluate a system at selected points.

\item The set $\left\{1, \cos(t), \sin(t) - \frac{2}{\pi}\right\}$ is an orthogonal basis for $\Span \ S$.

\ea

\item \label{ex:6_e_Legendre} $1$, $t$, $-\frac{1}{3} + t^2$, $t^3-\frac{3}{5}t$

\item 
\ba
\item An orthogonal basis for $V$ containing 1 is 
\[\left\{1, \frac{1}{3}+t,  \frac{1}{2} + \frac{1}{2}t + t^2\right\}.\]

\item $\proj_{\pol_1} (1+2t+3t^2) = \frac{1}{2}(-1+t)$

\ea

\oee

\be
\item[9.]
\begin{enumerate}[label=(\alph*), leftmargin=1\parindent]	
\item F 
\skipitems{1}
\item T 
\skipitems{1}
\item T

\end{enumerate}

\ee

%37
\hspace{-0.25in} Section \ref{sec:linear_transformation}
\obe

\item Use properties of the definite integral from calculus. 

\item \label{ex:8_a_inverse_isomorphism} 
\ba
\item Show that $(S \circ T)(\vx) = \vx$ for all $\vx$ in $V$ and that $(T \circ S)(\vw) = \vw$ for all $\vw$ in $W$. 
\item Define $S : \pol_3 \to \M_{2 \times 2}$ by 
	\[S(a+bt+ct^2+dt^3) = \left[ \begin{array}{cc} a&b\\ c&d \end{array} \right].\] 
\item The property that $T^{-1}(\vw) = \vv$ whenever $T(\vv) = \vw$ is the key to this problem.
\ea 

\item \label{ex:8_a_LT_opposite} 
\ba
\item linear transformation, one-to-one, onto 

\item linear transformation, one-to-one, onto 

\ea


\item \label{ex:8_a_isomorphic_dimension} Use the fact that $T$ is one-to-one to show that $\CC$ is linearly independent, and that $T$ is onto to show that $\CC$ spans $W$.
	
\item  Yes, but the vector space needs to be infinite dimensional. 

\item  Show that $T(\vv) = \vw$ has at most one solution for each $\vw$ in $W$. 

\item 
\ba
\item Think of $T(V')$ as the range of a suitable restriction of $T$. 

\item  Use Exercise 12.  It is possible.  
 
\item \label{ex:8_a_rank_nullity} Let $\{\vv_1, \vv_2, \ldots, \vv_k\}$ be a basis for $\Ker(T)$. Extend this basis to a basis $\{\vv_1$, $\vv_2$, $\ldots$, $\vv_k$, $\vv_{k+1}$, $\ldots$, $\vv_n\}$ of $V$. Use this basis to find a basis for $\Range(T)$.
	
\item This follows from fact that coordinate transformations are linear.  

\ea

\oee

\be
\item[19.]
\begin{enumerate}[label=(\alph*), leftmargin=1\parindent]	
\item F 
\skipitems{1}
\item T
\skipitems{1}
\item F 
\skipitems{1}
\item T
\skipitems{1}
\item F

	\end{enumerate}


	
\ee

%38
\hspace{-0.25in} Section \ref{sec:transformation_matrix}
\obe

\item  \label{ex:8_b_matrix_transformation} 
\ba
\item  $r(t) = r_0(1) + r_1(t) + r_2(t_2)$ and $[r(t)]_{\CB} = \left[ \begin{array}{c} r_0\\r_1\\r_2 \end{array} \right]$. 

\item $T$ is a linear transformation
	
\item The coordinate mapping is a linear transformation. 
	
\item $[T(p_0(t))]_{\CC} = \left[ \begin{array}{c} 1\\0\\1\\0 \end{array} \right]$,  $[T(p_1(t))]_{\CC} = \left[ \begin{array}{c} 1\\1\\0\\0 \end{array} \right]$, $[T(p_2(t))]_{\CC} = \left[ \begin{array}{c} 0\\1\\0\\1 \end{array} \right]$


\item $\left[ \begin{array}{ccc} 1&1&0 \\ 0&1&1 \\ 1&0&0 \\ 0&0&1 \end{array} \right]$.

\item  $[T(1+t-t^2)]_{\CC} =  \left[ \begin{array}{r} 2\\0\\1\\-1 \end{array} \right]$, $T(1+t-t^2) = 2(1) + 0(t) + 1(t^2) + (-1)t^3 = 2+t^2-t^3$.

\item  $T(1+t-t^2) = 2 + t^2 - t^3$. 

\ea


\item \label{ex:8_b_composite} 
\ba
\item Use the linearity of both $S$ and $T$.  

\item True

\ea

\item \label{ex:8_b_Ker_Nul} $\vx \in \Ker(T)$ if and only if $[\vx]_{\CB} \in \Nul [T]_{\CB}^{\CC}$ T

\item \label{ex:8_b_range_matrix} 
\ba
\item Since $\vw$ is in the range of $T$, there is a vector $\vv$ so that $T(\vv) = \vw$. 

\item Since $\vy$ is in the range of $T'$, there exists a vector $\vx$ in $\R^n$ so that $T'(\vx) = \vy$. What is $[\vw]_{\CC}$? 

\item How do we tell if $T'$ is onto?  
\ea

\oee

\be
\item[9.]
\begin{enumerate}[label=(\alph*), leftmargin=1\parindent]	
\item F 
\skipitems{1}
\item T
\skipitems{1}
\item T
\skipitems{1}
\item T
\skipitems{1}
\item T
	\end{enumerate}


\ee

%39
\hspace{-0.25in} Section \ref{sec:transformations_eigenvalues}
\obe
\item 
\ba
\item Use properties of the derivative. 

\item $[T]_{\CS} = \left[ \begin{array}{rrr} -1&1&0 \\ 0&-2&2 \\ 0&0&-3 \end{array} \right]$.

\item $-1$, $-2$, and $-3$ with bases $\{1\}$,  $\{-1+t\}$, $\{1-2t+t^2\}$ 

\item $\CB = \{-1, -1+t, 1-2t+t^2\}$ 

\ea

 
\item \label{ex:8_c_derivative_operator} 
\ba
\item Use properties of differentiable functions. 

\item Use properties of the derivative.  

\item What is $D\left(e^{\lambda x}\right)$? 

\ea

\item 
\ba
\item Use properties of the matrix transpose. 
 
\item For which matrices is $T(A) = A$?  

\item When is it possible to have $A^{\tr} = \lambda A$?  

\ea
	
\oee

\be
\item[6.]
\begin{enumerate}[label=(\alph*), leftmargin=1\parindent]	
\item F 
\skipitems{1}
\item T
\skipitems{1}
\item T
\skipitems{1}
\item T
\end{enumerate}

\ee

%40
\hspace{-0.25in} Section \ref{sec:JCF}

\obe
\item 
\ba
\item  $\left[ \begin{array}{ccc} 12&1&0 \\ 0&12&1 \\ 0&0&12 \end{array} \right]$ 

\item $\left[ \begin{array}{ccc} 9&0&0 \\ 0&9&1 \\ 0&0&9 \end{array} \right]$

\item $\left[ \begin{array}{ccc} 12&0&0 \\ 0&6&0 \\ 0&0&6 \end{array} \right]$

\item $\left[ \begin{array}{cccc} 2&1&0&0 \\ 0&2&0&0 \\ 0&0&1&1\\ 0&0&0&1 \end{array} \right]$ 

\ea

 
\item The Jordan normal form is $\left[ \begin{array}{cccc} 2&1&0&0 \\ 0&2&1&0 \\ 0&0&2&0 \\ 0&0&0&4 \end{array} \right]$.

\item $\left[ \begin{array}{cc} \lambda_1 & 0 \\ 0& \lambda_2 \end{array} \right]$, $\left[ \begin{array}{cc} \lambda_1 & 1 \\ 0& \lambda_1 \end{array} \right]$, $\left[ \begin{array}{ccc} \lambda_1 & 0 & 0  \\ 0& \lambda_2 & 0 \\ 0 & 0 & \lambda_3 \end{array} \right]$, $\left[ \begin{array}{ccc} \lambda_1 & 1 & 0  \\ 0& \lambda_1 & 0 \\ 0 & 0 & \lambda_2 \end{array} \right]$, $\left[ \begin{array}{ccc} \lambda_1 & 1 & 0  \\ 0& \lambda_1 & 1 \\ 0 & 0 & \lambda_1 \end{array} \right]$, $\left[ \begin{array}{ccc} \lambda_1 & 1 & 0  \\ 0& \lambda_1 & 0 \\ 0 & 0 & \lambda_1 \end{array} \right]$. 

\item $\left[\begin{array}{rrrrrrrrrr}
4 & 1 & 0 & 0 & 0 & 0 & 0 & 0 & 0 & 0 \\
0 & 4 & 0 & 0 & 0 & 0 & 0 & 0 & 0 & 0 \\
0 & 0 & 2 & 0 & 0 & 0 & 0 & 0 & 0 & 0 \\
0 & 0 & 0 & 2 & 1 & 0 & 0 & 0 & 0 & 0 \\
0 & 0 & 0 & 0 & 2 & 0 & 0 & 0 & 0 & 0 \\
0 & 0 & 0 & 0 & 0 & 2 & 1 & 0 & 0 & 0 \\
0 & 0 & 0 & 0 & 0 & 0 & 2 & 1 & 0 & 0 \\
0 & 0 & 0 & 0 & 0 & 0 & 0 & 2 & 1 & 0 \\
0 & 0 & 0 & 0 & 0 & 0 & 0 & 0 & 2 & 1 \\
0 & 0 & 0 & 0 & 0 & 0 & 0 & 0 & 0 & 2
\end{array}\right]$

\item \label{ex:8_d_ge_independent} 
\ba
\item Show that $\vv_{1} = (A-\lambda I)^{k-1} \vv_k$ for each $k$ from $2$ to $p$. 
\item 
	\begin{enumerate}[i.]
	\item $\vv_p$ is not in $\Nul (A-\lambda I)^{p-1}$, so $(A-\lambda I)^{p-1}\vv_p  \neq \vzero$

\item The same process as in i. shows that $x_{p-1} = 0$.  

\item Keep repeating the process.  
	
	\end{enumerate}
\ea

	
\item 
\ba
\item $C = [\vu_2 \ \vu_1]$ with $\vu_1 = [0 \ 1]^{\tr}$ and $\vu_2 = (A-I_2)\vu_1 = [-1 \ -1]^{\tr}$


\item $J^k = \left[ \begin{array}{cc} 1&k \\ 0 &1 \end{array} \right]$, $A^k = \left[ \begin{array}{cc} 1+k&-k\\k&1-k \end{array} \right]$

\ea

\item Show that $(A - \lambda I)S(\vx) = \vzero$

\item  $\left[ \begin{array}{ccc} 0&0&0 \\ 1&0&0 \\ 0&0&0 \end{array} \right]$ and $  \left[ \begin{array}{ccc} 0&0&0 \\ 0&0&0 \\ 1&0&0 \end{array} \right]$

\item Use a projection onto a subspace. 

\item $\{2,-2t,t^2\}$ 
 
\item 
\ba
\item 
	\begin{enumerate}[i.]
	\item Calculate powers of $A$. 
	
	\item $(I_3-A)\left(I_3+A+A^2\right) = I_3$. 

	\end{enumerate}
	
\item Calculate 
\[(I-A)\left(I + A + \cdots + A^{m-1}\right).\]

\ea

\item If $A$ is nilpotent and $\vx$ is an eigenvector of $A$, what is $A^m \vx$ for every positive integer $m$? If $0$ is the only eigenvalue of $A$, what is the characteristic polynomial of $A$? 
	
\oee

\be
\item[25.]
\begin{enumerate}[label=(\alph*), leftmargin=1\parindent]	
\item F 
\skipitems{1}
\item T
\skipitems{1}
\item F
\skipitems{1}
\item T
\skipitems{1}
\item T
\skipitems{1}
\item F
\skipitems{1}
\item F

\end{enumerate}

\ee


\end{multicols}


